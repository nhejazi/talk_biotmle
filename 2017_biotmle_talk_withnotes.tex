\documentclass[12pt,t,handout]{beamer}
\usepackage{graphicx}
\setbeameroption{show notes}
\setbeamertemplate{note page}[plain]
\usepackage{listings}
\usepackage{datetime}
\usepackage{url}

%Bibliography
\usepackage{natbib}
\bibpunct{(}{)}{,}{a}{}{;}
\usepackage{bibentry}
\nobibliography*

\def\notescolors{1}
\input{header.tex}

%%%%%%%%%%%%%%%%%%%%%%%%%%%%%%%%%%%%%%%%%%%%%%%%%%%%%%%%%%%%%%%%%%%%%%
% end of header
%%%%%%%%%%%%%%%%%%%%%%%%%%%%%%%%%%%%%%%%%%%%%%%%%%%%%%%%%%%%%%%%%%%%%%

% title info
\title{Targeted Learning for Biomarker Discovery}
\author{\href{http://nimahejazi.org}{Nima Hejazi}}
\institute{Division of Biostatistics \\
           University of California, Berkeley \\
           \url{https://www.stat.berkeley.edu/~nhejazi}
          }
\date{
  \href{http://nimahejazi.org}
      {\tt \scriptsize \color{foreground} nimahejazi.org}
  \\[-4pt]
  \href{https://github.com/nhejazi}
      {\tt \scriptsize \color{foreground} github.com/nhejazi}
  \\[-4pt]
  \href{https://twitter.com/nshejazi}
      {\tt \scriptsize \color{foreground} @nshejazi}
}


\begin{document}

% title slide
{
\setbeamertemplate{footline}{} % no page number here
\frame{
  \titlepage

  \vfill \hfill \includegraphics[height=6mm]{Figs/cc-zero.png} \vspace*{-1cm}

  \note{These are slides for a talk given most recently at the Division of
    Biostatistics seminar series, at the University of California, Berkeley on
    13 March 2017.

    Source: {\tt https://github.com/nhejazi/talk-biotmle-2017} \\
    Slides: {\tt http://bit.ly/CBB2016\_nonotes} \\
    With notes: {\tt http://bit.ly/CBB2016\_wnotes}
}
}
}


\begin{frame}[c]{Preview: Summary}
\only<1>{\addtocounter{framenumber}{-1}}

  \begin{enumerate}
  	\itemsep12pt
  	\item We've developed a protocol for analyzing RRBS data.
  	\item TMLE provides a robust framework for the analysis of
          miRNA data, comparing favorably against LIMMA.
  	\item The data-adaptive multiple testing approach developed
          can successfully be applied to screening (based on
          variable importance measures).
  	\item A TMLE-based technique for estimation/inference of DNA
          methylation (in particular, 450K data) has been
          formulated, and an R package is in development.
  	\item Combining data-adaptive multiple testing with the above
    	  TMLE method promises to provide a robust framework for
          analyzing DNA methylation data.
  \end{enumerate}

  \note{It's always good to include a summary.}

\end{frame}


\begin{frame}[fragile,c]{}

\begin{center}
\begin{minipage}[c]{9.3cm}
\begin{semiverbatim}
\lstset{basicstyle=\normalsize}
\begin{lstlisting}[linewidth=9.3cm]
 Karl -- this is very interesting,
 however you used an old version of
 the data (n=143 rather than n=226).

 I'm really sorry you did all that
 work on the incomplete dataset.

 Bruce
\end{lstlisting}
\end{semiverbatim}
\end{minipage}
\end{center}

\note{This is an edited version of an email I got from a collaborator,
  in response to an analysis report that I had sent him.

  I try to always include some brief data summaries at the start of
  such reports. By doing so, he immediately saw that I had an old
  version of the data.

  Because I'd set things up carefully, I could just substitute in the
  newer dataset, type ``{\tt make}'', and get the revised report.

  This is a reproducibility success story. But it took me a long
  time to get to this point.
}
\end{frame}


\begin{frame}[c]{Analyzing RRBS Data, I}

\begin{center}
\begin{itemize}
	\item ...
  \item ...
\end{itemize}
\end{center}

\note{}

\end{frame}


\begin{frame}[c]{Analyzing RRBS Data, II}

\begin{center}
\begin{itemize}
	\item ...
  \item ...
\end{itemize}
\end{center}

\note{}

\end{frame}


\begin{frame}[c]{Analyzing RRBS Data, III}

\begin{center}
\begin{itemize}
	\item ...
    \item ...
\end{itemize}
\end{center}

\note{}

\end{frame}


\begin{frame}[c]{Analyzing RRBS Data, I}

\begin{center}
\begin{itemize}
	\item ...
    \item ...
\end{itemize}
\end{center}

\note{}

\end{frame}


\begin{frame}[c]{Analyzing RRBS Data, II}

\begin{center}
\begin{itemize}
	\item ...
  \item ...
\end{itemize}
\end{center}

\note{}

\end{frame}


\begin{frame}[c]{Review: Summary}

  \begin{enumerate}
  	\itemsep12pt
  	\item We've developed a protocol for analyzing RRBS data.
  	\item TMLE provides a robust framework for the analysis of
          miRNA data, comparing favorably against LIMMA.
  	\item The data-adaptive multiple testing approach developed
          can successfully be applied to screening (based on
          variable importance measures).
  	\item A TMLE-based technique for estimation/inference of DNA
          methylation (in particular, 450K data) has been
          formulated, and an R package is in development.
  	\item Combining data-adaptive multiple testing with the above
    	  TMLE method promises to provide a robust framework for
          analyzing DNA methylation data.
  \end{enumerate}

  \note{It's always good to include a summary.}

\end{frame}


\begin{frame}[c]{}

\Large
Slides: \href{http://bit.ly/biotmle2017_wnotes}{link} \quad
\includegraphics[height=5mm]{Figs/cc-zero.png}

\vspace{10mm}

\href{http://nimahejazi.org}{\tt nimahejazi.org}

\vspace{10mm}

\href{https://github.com/nhejazi}{\tt github.com/nhejazi}

\vspace{10mm}

\href{https://twitter.com/nshejazi}{\tt @nshejazi}


\note{
  Here's where you can find me, as well as the slides for this talk.
}
\end{frame}


\end{document}
