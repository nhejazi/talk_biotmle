\documentclass[12pt,t]{beamer}
\usepackage{graphicx}
\setbeameroption{hide notes}
\setbeamertemplate{note page}[plain]
\usepackage{listings}
\usepackage{datetime}
\usepackage{url}

%Bibliography
\usepackage{natbib}
\bibpunct{(}{)}{,}{a}{}{;}
\usepackage{bibentry}
\nobibliography*

\input{header.tex}

%%%%%%%%%%%%%%%%%%%%%%%%%%%%%%%%%%%%%%%%%%%%%%%%%%%%%%%%%%%%%%%%%%%%%%
% end of header
%%%%%%%%%%%%%%%%%%%%%%%%%%%%%%%%%%%%%%%%%%%%%%%%%%%%%%%%%%%%%%%%%%%%%%

% title info
\title{Empirical Bayes Moderation of Asymptotically Linear Parameters}
\author{\href{http://nimahejazi.org}{Nima Hejazi}}
\institute{Division of Biostatistics \\
           University of California, Berkeley \\
           \href{https://www.stat.berkeley.edu/~nhejazi}
             {\tt \scriptsize \color{foreground} stat.berkeley.edu/\textasciitilde{}nhejazi}
          }
\date{
  \href{http://nimahejazi.org}
      {\tt \scriptsize \color{foreground} nimahejazi.org}
  \\[-4pt]
  \href{https://twitter.com/nshejazi}
      {\tt \scriptsize \color{foreground} twitter/@nshejazi}
  \\[-4pt]
  \href{https://github.com/nhejazi}
      {\tt \scriptsize \color{foreground} github/nhejazi}
}


\begin{document}

% title slide
{
\setbeamertemplate{footline}{} % no page number here
\frame{
  \titlepage

  \vspace{-2em}

  \centerline{\href{https://goo.gl/6ou8YR}{\tt \scriptsize \underline{slides}: goo.gl/6ou8YR}}

  \vfill \hfill \includegraphics[height=6mm]{Figs/cc-zero.png} \vspace*{-0.75cm}

  \note{These are slides for a talk given most recently at the Division of
    Biostatistics seminar series, at the University of California, Berkeley on
    13 March 2017.

    Source: {\tt https://github.com/nhejazi/talk-biotmle} \\
    Slides: {\tt https://goo.gl/r3zsu6} \\
    With notes: {\tt https://goo.gl/6ou8YR}
}
}
}



\begin{frame}[c]{Preview: Summary}
\only<1>{\addtocounter{framenumber}{-1}}

\begin{center}
\begin{itemize}
  \itemsep12pt
  \item TMLE provides a robust framework for the analysis of miRNA data,
    comparing favorably against LIMMA.
  \item The data-adaptive multiple testing approach developed can successfully
    be applied to screening (based on variable importance measures).
  \item A TMLE-based technique for estimation/inference of DNA methylation
    (in particular, 450K data) has been formulated, and an R package is in
    development.
  \item Combining data-adaptive multiple testing with the above TMLE method
    promises to provide a robust framework for analyzing DNA methylation data.
\end{itemize}
\end{center}


\note{We'll go over this summary again at the end of the talk. Hopefully, it
  will all make more sense then.
}

\end{frame}



\begin{frame}[c]{Targeted Minimum Loss-Based Estimation (TMLE)}

\begin{center}
\begin{itemize}
  \item TMLE produces a well-defined, unbiased, efficient substitution estimator
    of target parameters of a data-generating distribution.
  \item It is generally an iterative procedure (though there is now a general
    one-step) that updates an initial (super learner) estimate  of the relevant
    part $Q_0$ of the data generating distribution $P_0$, possibly using an
    estimate of a nuisance parameter $g_0$.
  \item Like corresponding A-IPTW estimators, removes asymptotic residual bias
    of initial estimator for the target parameter, if it uses a consistent
    estimator of $g_0$, thus {\em Doubly Robust}.
  \item If initial estimator was consistent for the target parameter, the
    additional fitting of the data in the targeting step may remove finite
    sample bias, and  preserves consistency property of the initial estimator.
\end{itemize}
\end{center}

\note{...}

\end{frame}



\begin{frame}[c]{TMLE via Machine Learning}

\begin{center}
\begin{itemize}
  \item Natural use of machine learning methods for the estimation of both $Q_0$
    and $g_0$.
  \item Focuses effort to achieve minimal bias and asymptotic semi-parametric
    efficiency bound for the variance, but still get inference (with some
    assumptions).
\end{itemize}
\end{center}

\note{...}

\end{frame}



\begin{frame}[c]{Inference (Standard errors) via Influence Curve (IC)}

\begin{center}
\begin{itemize}
\item influence curve\index{influence curve} for estimator is:
\begin{multline*}
IC_n(O_i)=\left (\frac{I(A_i=a_h)}{g_n(a_h\mid W_i)}-\frac{I(A_i=a_l)}{g_n(a_l\mid W_i)}\right)(Y-\bar{Q}^1_n(A_i,W_i))\\
+\bar{Q}^1_n(a_h,W_i)-\bar{Q}^1_n(a_l,W_i)-\psi_{TMLE,n},
\end{multline*}
\item Sample variance of the estimated influence curve:
\[\textstyle S^2(IC_n)=\frac{1}{n}\sum_{i=1}^n\left(IC_n(o_i)\right)^2.\]
\item Use sample variance to estimate the standard error of our estimator:
\[SE_n=\sqrt{\frac{S^2(IC_n)}{n}}.\]
\item Use this to derive uncertainty measures (p-values, confidence intervals, etc.).
\end{itemize}
\end{center}

\note{...}

\end{frame}



\begin{frame}[c]{Repeating Estimates of Variable Importance one biomarker at a
  time}

\Large{
\begin{itemize}
\item Consider this is repeated for $j=1,\ldots,J$  different biomarkers, so that one has, for each $j$: 
$$\Psi_j(Q_{j,n}^{*}),S_j^2(IC_{j,n})$$
or estimate of variable importance and standard error for all $J$.
\item Propose an existing joint-inferential procedure that can add some finite-sample robustness to an estimator that can be highly variable.
\end{itemize}
}

\note{...}

\end{frame}



\begin{frame}[c]{LIMMA: Linear Models for Microarray Data}

\begin{center}
\begin{itemize}
\item Thus, one can define a standard t-test statistic for a general (asymptotically linear) parameter estimate (over $g=1,\ldots,G$) as:
\[
t_g = \frac{\sqrt{n}(\Psi_g(P_n)-\psi_0)}{S_g(IC_{g,n})}
\]
\item Consider the moderated t-statistic proposed by \cite{Smyth:2005qy}:
\[
\tilde{t}_g = \frac{\sqrt{n}(\Psi_g(P_n)-\psi_0)}{\tilde{S}_g^2}
\]
where the posterior estimate of the variance of the influence curve is  
\[
\tilde{S}^2_g = \frac{d_0 S^2_0 +d_g S^2_g(IC_{g,n})}{d_0+d_g}
\]
\end{itemize}
\end{center}

\note{...}

\end{frame}



\begin{frame}[c]{Implement for Any Asymptotically Linear Parameter}

\begin{center}
\begin{itemize}
\item Treat like one-sample problem estimate of parameter with associated SE from IC.
\item Just need to get estimate for each $g$ as well as the plug-in IC for every observation for that $g$ and repeat for all $g$.
\item Transform data original $G x n$ matrix where new entries are:
\[
Y^*_{g,i} = IC_{g,n}(O_i; P_n)+\Psi_g(P_n)
\]
\item Since the average of the $IC_{g,n}$ across the columns (units) for a single $g$ will be 0, the average of this transform will be the original estimate $\Psi_g(P_n)$.
\item For simplicity assume the null value is $\psi_0 = 0$ for all $g$.  Then, running {\em limma} package on this transform,$Y^*_{g,i}$,  will generate multiple testing corrections based on presented above for $\tilde{t}_g$.
\end{itemize}
\end{center}

\note{...}

\end{frame}



\begin{frame}[c]{Why LIMMA approach in this context?}

\begin{center}
\begin{itemize}
\item Often times these analyses based on relatively small samples.
\item want data-adaptive estimate but at least with standard implementation of these estimates (estimation equation, substitution,...), the SE's can be non-robust.
\item practically, one can get "significant" estimates of variable importance measures that are driven by poorly and underestimated $S^2_g(IC_{g,n})$. 
\item LIMMA shrinks these $S^2_g(IC_{g,n})$ by making them bigger and thus takes biomarkers with small parameter estimates but very small $S^2_g(IC_{g,n})$ out of statistical significance.
\item Also, just seems to work very well...
\end{itemize}
\end{center}

\note{...}

\end{frame}



\begin{frame}[c]{Data Analysis I}

\begin{center}
\begin{itemize}
  \itemsep12pt
  \item $\dots$
  \item $\dots$
  \item $\dots$
  \item $\dots$
\end{itemize}
\end{center}

\note{...}

\end{frame}



\begin{frame}[c]{Data Analysis II}

\begin{center}
\begin{itemize}
  \itemsep12pt
  \item $\dots$
  \item $\dots$
  \item $\dots$
  \item $\dots$
\end{itemize}
\end{center}

\note{...}

\end{frame}



\begin{frame}[c]{Data Analysis III}

\begin{center}
\begin{itemize}
  \itemsep12pt
  \item $\dots$
  \item $\dots$
  \item $\dots$
  \item $\dots$
\end{itemize}
\end{center}

\note{...}

\end{frame}



\begin{frame}[c]{Review: Summary}

\begin{center}
\begin{itemize}
  \itemsep12pt
  \item TMLE provides a robust framework for the analysis of miRNA data,
    comparing favorably against LIMMA.
  \item The data-adaptive multiple testing approach developed can successfully
    be applied to screening (based on variable importance measures).
  \item A TMLE-based technique for estimation/inference of DNA methylation
    (in particular, 450K data) has been formulated, and an R package is in
    development.
  \item Combining data-adaptive multiple testing with the above TMLE method
    promises to provide a robust framework for analyzing DNA methylation data.
\end{itemize}
\end{center}

\note{It's always good to include a summary.}

\end{frame}


\begin{frame}{Acknowledgments}

\vspace{18pt}

\begin{tabular}{@{}l@{\hspace{1.5cm}}l@{}}
Alan Hubbard & \footnotesize \lolit University of CA, Berkeley \\
\\[0.5ex]
Mark van der Laan & \footnotesize \lolit University of CA, Berkeley \\

%\\[2ex]

%Collaborator, the first & \footnotesize \lolit University or Institution 2 \\
%Collaborator, the second & \footnotesize \lolit University or Institution 2 \\
\end{tabular}

\vspace{10mm}

%NIH - funding source ?

\note{}

\end{frame}



\begin{frame}[c]{}

\Large
Slides: \href{https://goo.gl/6ou8YR}{goo.gl/6ou8YR} \quad
\includegraphics[height=5mm]{Figs/cc-zero.png}

\vspace{10mm}

\href{https://www.stat.berkeley.edu/~nhejazi}{\tt stat.berkeley.edu/\textasciitilde{}nhejazi}

\vspace{10mm}

\href{http://nimahejazi.org}{\tt nimahejazi.org}

\vspace{10mm}

\href{https://twitter.com/nshejazi}{\tt twitter/@nshejazi}

\vspace{10mm}

\href{https://github.com/nhejazi}{\tt github/nhejazi}


\note{Here's where you can find me, as well as the slides for this talk.}

\end{frame}



\end{document}
